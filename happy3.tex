%iffalse
\let\negmedspace\undefined
\let\negthickspace\undefined
\documentclass[journal,12pt,twocolumn]{IEEEtran}
\usepackage{cite}
\usepackage{amsmath,amssymb,amsfonts,amsthm}
\usepackage{algorithmic}
\usepackage{graphicx}
\usepackage{textcomp}
\usepackage{xcolor}
\usepackage{txfonts}
\usepackage{listings}
\usepackage{enumitem}
\usepackage{mathtools}
\usepackage{gensymb}
\usepackage{comment}
\usepackage[breaklinks=true]{hyperref}
\usepackage{tkz-euclide} 
\usepackage{listings}
\usepackage{gvv}                                        
%\def\inputGnumericTable{}                                 
\usepackage[latin1]{inputenc}                                
\usepackage{color}                                            
\usepackage{array}                                            
\usepackage{longtable}                                       
\usepackage{calc}                                             
\usepackage{multirow}                                         
\usepackage{hhline}                                           
\usepackage{ifthen}                                           
\usepackage{lscape}
\usepackage{tabularx}
\usepackage{array}
\usepackage{float}


\newtheorem{theorem}{Theorem}[section]
\newtheorem{problem}{Problem}
\newtheorem{proposition}{Proposition}[section]
\newtheorem{lemma}{Lemma}[section]
\newtheorem{corollary}[theorem]{Corollary}
\newtheorem{example}{Example}[section]
\newtheorem{definition}[problem]{Definition}
\newcommand{\BEQA}{\begin{eqnarray}}
\newcommand{\EEQA}{\end{eqnarray}}
\newcommand{\define}{\stackrel{\triangle}{=}}
\theoremstyle{remark}
\newtheorem{rem}{Remark}

% Marks the beginning of the document
\begin{document}
\bibliographystyle{IEEEtran}
\vspace{3cm}

\title{CHAPTER - 19\\Differential Equations}
\author{EE24BTECH11039 - Ranjith}
\maketitle
\newpage
\bigskip

\renewcommand{\thefigure}{\theenumi}
\renewcommand{\thetable}{\theenumi}


\section{MCQ's with One Correct Answer}

\begin{enumerate}
\item The differential equation whose solution is $Ax^2 + By^2 = 1$ where A and B are arbitrary constants is of 



\begin{enumerate}
\item second order and second degree 
\item first order and second degree 
\item first order and first degree 
\item second order and first degree 

\end{enumerate}
\hfill
{[2006]}
\item The differential equations of all the circles passing through the origin and having their centres on x-axis is

\begin{enumerate}
\item $ y^2=x^2 + 2xy\diff{y}{x} $
\item $ y^2=x^2 - 2xy\diff{y}{x}$
\item $ x^2=y^2 + xy\diff{y}{x}$
\item $ x^2 =y^2 + 3xy\diff{y}{x}$
\end{enumerate}
\hfill
{{[2007]}}




\item The solution of the differential equation $ \diff{y}{x}=\frac{x+y}{x}$ satisfying the condition $ y(1)=1 $ is
\begin{enumerate}
\item $ y=  \ln{x}+x $
\item $y=x\ln{x}+x^2$
\item $ y=xe^{(x-1)} $
\item $ y=x \ln{x}+ x$
\end{enumerate}
\hfill
{[2008]}


\item The differential equation which represents the family of curves $y= c{_1}e^{c_2}x , where c{_1} and c{_2}$ are arbitrary constants,is
\begin{enumerate}
    \item $\frac{d^{2}y}{dx^{2}}$= y\diff{y}{x}

    
    \item $y\frac{d^{2}y}{dx^{2}}$=\diff{y}{x} 

    
    \item $y\frac{d^{2}y}{dx^{2}}=(\diff{y}{x})^2$
    
    \item $\diff{y}{x}=y^2$
\end{enumerate}
\hfill
{[2009]}
\item Solutions of the differential equation $\cos{x} d{y}=y(\sin{x}-y)d{x},0<x<\frac{\pi}{2}$ is
\begin{enumerate}

\item $ y\sec{x}=\tan{x}+c$
\item $y\tan{x}=\sec{x}+c$
\item $\tan{x}=(\sec{x}+c)y$
\item $\sec{x}=(\tan{x}+c)y$
\hfill
{{[2010]}}




    
\end{enumerate}

\item If $\frac{d^{2}y}{dx^{2}}=y+3$ and $y(0)=2$, then $y(\ln{2}$ is equal to:
\begin{enumerate}
\item $ 5 $
\item $ 13 $
\item $ -2 $
\item $ 7 $
\end {enumerate}

\hfill
{{[2011]}}



\item Let  be the purchase value of an equipment and $V(t)$ be the value after it has been used for t years. The value V(t) depreciates at a rate given by differential equation $\diff{V_{(t})}{t}=-k(T-t).$ where k is a constant and T is the total life in years of the equipment.Then the scrap value $V(T)$ of the equipment is 

\begin {enumerate}

\item $ l -\frac{kT^2}{2}$
\item $ l - \frac{k(T-t)^2}{2}$
\item $ e^{-kT}$
\item $ T^2-\frac{1}{k}$

\end{enumerate}
\hfill
{[2011]}

\item The population $p(t)$ at time of a certain mouse species satisfies the differential equation $\diff{p{_(t)}}{t}= 0.5p(t)- 450$.If $p(0)=850$,then the time at which the population becomes zero is:
\begin{enumerate}
    \item $ 2\ln{18}$
    \item $ 2\ln{9}$
    \item $ \frac{1}{2}\ln{18}$
\end{enumerate}
\hfill
{[2012]}

\item At present , a firm is manufacturing 2000times.It is estimated that the rate of change of production P with respect to additional number of workers $x$ is given by $\diff{P}{x}=100-12\sqrt{x}$.If the firm employs 25 more workers,then the new level of production of items is


\begin{enumerate}

    \item $2500$
    \item$3000$
    \item$3500$
    \item$4500$
\end{enumerate}

\hfill
{[JEE M 2013]}


\item Let the population of rabbits surviving at time t be governed by the differential equation $\diff{p{_(t)}}{t}=\frac{1}{2}p(t)-200$.If$P(0)=100,then p(t)equals$:
\begin{enumerate}
    \item $ 600-500e^\frac{t}{2}$
    \item $ 400-300e^\frac{-t}{2}$
    \item $ 400-300e^\frac{t}{2}$
    \item $ 300-200e^\frac{-t}{2}$
\end{enumerate}
\hfill
{[JEE M 2014]}
\item Let $y(x)$ be the solution of the differential equation $(x\log{x})\diff{y}{x} + y = 2x \log{x},(x\geq 1)$.Then $y(e)$ is equal to:
\begin{enumerate}
    \item $ 2 $
    \item $ 2e $
    \item $ e $
    \item $ 0 $
\end{enumerate}
\hfill
{[JEE M 2015]}

\item If the curve $ y=f(x)$ passes through the point $ (1,1)$ and satisfies the differential equation,$ y(1+xy)d{x}=xd{y},$ then $ f(\frac{-1}{2})$ is equals to 
\begin{enumerate}
\item $ \frac{2}{5}$



\item $ \frac{4}{5}$



\item $ \frac{2}{5}$



\item $ \frac{4}{5}$


\end{enumerate}

\hfill
{[JEE M 2016]}

\item If $ (24 \sin{x})\diff{y}{x}+ (y+1)\cos{x}=0$ and $y(0)=1$ then $y(\frac{\pi}{2})$ is equal to
\begin{enumerate}
    \item $\frac{4}{3}$
    \item $\frac{1}{3}$
    \item $\frac{2}{3}$
    \item $\frac{1}{3}$
    
\end{enumerate}
\hfill
{[JEE M 2017]}


\item Let $ y=y(x)$ be the solution of the differential equation $\sin{x}\diff{y}{x}+y\cos{x}=4x$,$x\in(0,2)$.If $y(\frac{\pi}{2})=0$,then $y(\frac{\pi}{6})$ is equal to
\begin{enumerate}
    \item $ \frac{-8}{9\sqrt{3}}\pi^2$
    \item $ \frac{-8}{9}\pi^2$
    \item $\frac{-4}{9}\pi^2$
    \item $ \frac{4}{9\sqrt{3}}\pi^2 $
\end{enumerate}
\hfill
{[JEE M 2018]}
\item If $ y=y(x)$ is the differential equation $ \sin{x}\diff{y}{x}+2y=x^2$ satisfying $y(a)=1$,then $y(\frac{1}{2})$ is equal to
\begin{enumerate}
    \item $ \frac{7}{64}$

    
    \item $ \frac{1}{4}$

    
    \item $ \frac{49}{16}$
    
    \item $ \frac{13}{16}$
    
    
\end{enumerate}
\hfill
{[JEE M 2019-9April(M)]}
\item The solution of the differential equation $ x\diff{y}{x}+2y=x^2 (x\neq0)$with$ y(1)=1,is:$
\begin{enumerate}
    \item $ y=\frac{4}{5}x^3+\frac{1}{5x^2}$

\item $ y=\frac{x^3}{5}+\frac{1}{5x^2}$
\item $ y=\frac{x^2}{4}+\frac{3}{4x^2}$
\item $ y=\frac{3}{4}x^2+\frac{1}{4x^2}$
\end{enumerate}
\hfill
{[JEE M 2019-9April(M)]}















\end{enumerate}







































\end{document}
