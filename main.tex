%iffalse
\let\negmedspace\undefined
\let\negthickspace\undefined
\documentclass[journal,12pt,twocolumn]{IEEEtran}
\usepackage{cite}
\usepackage{amsmath,amssymb,amsfonts,amsthm}
\usepackage{algorithmic}
\usepackage{graphicx}
\usepackage{textcomp}
\usepackage{xcolor}
\usepackage{txfonts}
\usepackage{listings}
\usepackage{enumitem}
\usepackage{mathtools}
\usepackage{gensymb}
\usepackage{comment}
\usepackage[breaklinks=true]{hyperref}
\usepackage{tkz-euclide} 
\usepackage{listings}
\usepackage{gvv}                                        
%\def\inputGnumericTable{}                                 
\usepackage[latin1]{inputenc}                                
\usepackage{color}                                            
\usepackage{array}                                            
\usepackage{longtable}                                       
\usepackage{calc}                                             
\usepackage{multirow}                                         
\usepackage{hhline}                                           
\usepackage{ifthen}                                           
\usepackage{lscape}
\usepackage{tabularx}
\usepackage{array}
\usepackage{float}


\newtheorem{theorem}{Theorem}[section]
\newtheorem{problem}{Problem}
\newtheorem{proposition}{Proposition}[section]
\newtheorem{lemma}{Lemma}[section]
\newtheorem{corollary}[theorem]{Corollary}
\newtheorem{example}{Example}[section]
\newtheorem{definition}[problem]{Definition}
\newcommand{\BEQA}{\begin{eqnarray}}
\newcommand{\EEQA}{\end{eqnarray}}
\newcommand{\define}{\stackrel{\triangle}{=}}
\theoremstyle{remark}
\newtheorem{rem}{Remark}

% Marks the beginning of the document
\begin{document}
\bibliographystyle{IEEEtran}
\vspace{3cm}

\title{CHAPTER - 10\\Functions}
\author{EE24BTECH11039 - Ranjith}
\maketitle
\newpage
\bigskip

\renewcommand{\thefigure}{\theenumi}
\renewcommand{\thetable}{\theenumi}

\fontsize{18}
{20}\selectfont
\section{A: Fill in the Blanks}


\begin{enumerate}
\item The values of $f(x)=3\sin\brak{\sqrt{\frac{\pi^2}{16}-x^2}} $ lie in the interval .........


\hfill{(1983 - 1 Mark)}


 \item For the function $f(x) = \begin{cases} \frac{x}{1 + e^{1/x}}, & x \neq 0 \\ 0, & x = 0 \end{cases} $
the derivative from the right, $f'(0+)$ =..... , and the derivative from the left, $f'(0-)$ =......


 
\hfill{(1983 - 2mark)}

\item{The domain of the function $f(x) =\sin^{-1}\left(\log_{2}\left(\frac{x^{2}}{2}\right)\right)$ is given by \ldots \


\hfill 
{(1984 - 2mark)}

\item Let \( A \) be a set of \( n \) distinct elements. Then the total number of distinct functions from \( A \) to \( A \) is \( \underline{\hspace{2cm}} \) and out of these \( \underline{\hspace{2cm}} \) are onto functions.

\hfill
{(1985- 2mark)}


\item If $f(x) = \sin \left[ \ln \left( \frac{\sqrt{4 - x^{2}}}{1 - x} \right) \right]$, { then domain of } f(x) \text{ is ... and its} \\ \text{range is .........}


\hfill
{(1985 - 2Mark)}

 
\item There are exactly two distinct linear functions,...and...which map [-1,1]onto [0,2]

\hfill
{(1989 - 1Mark)}



 \item If f is a even function defined on the 
interval (-5,5),then four real values of x 
satisfying the equation $f(x)=f({\frac{x+2}{x+1}})$
are.......... and.......


\hfill{(1996 - 1mark)}
}
\end{enumerate}



\fontsize{18}
{20}\selectfont
\section{B: True / False}




\begin{enumerate}

\item If $f(x)=(a-x^n)^{1/n}$ where $a>0$ n is a positive integer 
then $f(f(x))=x.$


 \hfill
 {(1983 - 1Mark)}


 
\item The function $f(x)={\frac{x^2+4x+30}{x^2-8x+18}}$ is not one-to one.


\hfill
{(1983 - 1Mark)}




\item If $f{_1}(x)$
 and  $f{_2}(x)$ are  defined on domains $D{_1} and D{_2}$ respectively, then $f{_1}(x)$ + $f{_2}(x)$ is defined on $D{_1}\cup D{_2}$.


\hfill
{(1988 - 1Mark)}
\end{enumerate}


\fontsize{18}
{20}\selectfont
\section{C: MCQ's with One Correct Answer}



 


\begin{enumerate}
     


\item Let R be the set of real numbers.If $f:R \mapsto R $ is
a function defined by$f(x)= x^2 $,then f is:

\begin{enumerate}
    
\item Injective but not surgective 
 
\item Surjective but not injective
 
\item Bijective
 
\item None of these.
\end{enumerate}


\hfill
(1987)


 


\item The entire graphs of the equation $y= x{^2}+ kx - x +9$ is strictly above the x-axis 
if and only if
\begin{enumerate}
    


 \item $k<7$
 
 \item $-5<k<7$
 
 \item $k>-5$
 
 \item None of these.
 \end{enumerate}
 \hfill
 (1979)


\item Let $f(x)=|x-1|.$then

\begin{enumerate}
  \item $f(x^2)$ = $(f(x))^2$

  \item  $f(x+y)=f(x)+f(y)$

  \item $f(|x|)=|f(x)|$

  \item None of these.
  \end{enumerate}
  
  \hfill
  (1983 - 1Mark)




  \item If x satisfies $|x-1| + |x-2| + |x-3|\geq6$, then

\begin{enumerate}
    

  \item $0\leq x\leq4$
  
  \item $x \leq-2$ or $x\geq4$
  
  \item $x\leq0$ or$x\geq4$
  
  \item None of these.

  \end{enumerate}
  \hfill
  (1983-1Mark)



  

\end{enumerate}

\end{document}


