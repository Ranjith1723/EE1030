\documentclass[journal]{IEEEtran}
\usepackage[a5paper, margin=10mm]{geometry}
%\usepackage{lmodern} % Ensure lmodern is loaded for pdflatex
\usepackage{tfrupee} % Include tfrupee package


\setlength{\headheight}{1cm} % Set the height of the header box
\setlength{\headsep}{0mm}     % Set the distance between the header box and the top of the text


%\usepackage[a5paper, top=10mm, bottom=10mm, left=10mm, right=10mm]{geometry}

%
\usepackage{gvv-book}
\usepackage{gvv}
\setlength{\intextsep}{10pt} % Space between text and floats

\makeindex

\begin{document}
\bibliographystyle{IEEEtran}
\onecolumn

\title{CHAPTER - 10\\Functions}
\author{EE24BTECH11039 - Ranjith}
\maketitle


\renewcommand{\thefigure}{\theenumi}
\renewcommand{\thetable}{\theenumi}


\section{ Fill in the Blanks}


\begin{enumerate}
    \item The values of \begin{align*}f\brak{x}=3\sin\brak{\sqrt{\frac{\pi^2}{16}-x^2}}\end{align*}lie in the interval .........
    
    
    \hfill{(1983 - 1 Mark)}
    
    
     \item For the function \begin{align*}f\brak{x} = \begin{cases} \frac{x}{1 + e^{1/x}}, & x \neq 0 \\ 0, & x = 0 \end{cases} \end{align*}
    the derivative from the right, $f^{\prime}\brak{0+} $=..... , and the derivative from the left, $f^{\prime }\brak{0-}$=......
    
    
     
    \hfill{(1983 - 2mark)}
    
    \item{The domain of the function $f\brak{x}=\sin^{-1}\left(\log_{2}\left(\frac{x^{2}}{2}\right)\right)$ is given by \ldots \
    
    
    \hfill 
    {(1984 - 2mark)}
    
    \item Let $A$ be a set of $n$ distinct elements. Then the total number of distinct functions from $ A $ to $ A $ is \( \underline{\hspace{2cm}} \) and out of these \( \underline{\hspace{2cm}} \) are onto functions.
    
    \hfill
    {(1985- 2mark)}
    
    
    \item If \begin{align*}  f\brak{x} = \sin \brak{ \ln \brak{ \frac{\sqrt{4 - x^{2}}}{1 - x}}} \end{align*},  then domain of  $f\brak{x}$ is ... and its range is .........
    
    
    \hfill
    {(1985 - 2Mark)}
    
     
    \item There are exactly two distinct linear functions,...and...which map $[-1,1]onto [0,2]$
    
    \hfill
    {(1989 - 1Mark)}
    
    
    
     \item If f is a even function defined on the 
    interval $\brak{-5,5}$,then four real values of $x $
    satisfying the equation $f\brak{x}=f\brak{{\frac{x+2}{x+1}}}$
    are.......... and.......
    
    
    \hfill   (1996 - 1mark)
    }
\end{enumerate}





\section{ True / False}




\begin{enumerate}

\item If  $f\brak{x}=(a-x^n)^{1/n}$where $a>0$ n is a positive integer 
then $f\brak{f\brak{x}}=x.$


 \hfill 
 {(1983 - 1Mark)}


 
\item The function $f\brak{x}={\frac{x^2+4x+30}{x^2-8x+18}}$ is not one-to one.


\hfill
{(1983 - 1Mark)}




\item If $f{_1}\brak{x}$
 and  $f{_2}\brak{x}$ are  defined on domains $D{_1} and D{_2}$ respectively, then $f{_1}\brak{x}$ + $f{_2}\brak{x}$ is defined on $D{_1}\cup D{_2}$.


\hfill
{(1988 - 1Mark)}
\end{enumerate}




\section{ MCQ's with One Correct Answer}



 


\begin{enumerate}
     


  \item Let R be the set of real numbers.If $f:R \mapsto R $ is
  a function defined by$f\brak{x}= x^2 $,then f is:

    \begin{multicols}{2}
      \begin{enumerate}
          
        \item Injective but not surgective 
        
        \item Surjective but not injective
        
        \item Bijective
        
        \item None of these.
      \end{enumerate}
    \end{multicols}

  \hfill
  (1987)


 


  \item The entire graphs of the equation $y= x{^2}+ kx - x +9$ is strictly above the x-axis 
  if and only if
    \begin{multicols}{2}
      \begin{enumerate}
          


        \item $k<7$
        
        \item $-5<k<7$
        
        \item $k>-5$
        
        \item None of these.
      \end{enumerate}
    \end{multicols}
  \hfill
  (1979)


\item Let $f\brak{x}=|x-1|.$then
  \begin{multicols}{2}
      \begin{enumerate}
      \item $f\brak{x^2}$ = $(\brak{x})^2$

      \item  $f\brak{x+y}=f\brak{x}+f\brak{y}$

      \item $f\brak{|x|}=|f\brak{x}|$

      \item None of these.
      \end{enumerate}
  \end{multicols}
  \hfill
  (1983 - 1Mark)



  \item If f\brak{x} satisfies $|x-1| + |x-2| + |x-3|\geq6$, then
  \begin{multicols}{2}
    \begin{enumerate}
        

      \item $0\leq x\leq4$
      
      \item $x \leq-2$ or $x\geq4$
      
      \item $x\leq0$ or$x\geq4$
      
      \item None of these.

    \end{enumerate}
  \end{multicols}
  \hfill
  (1983-1Mark)



  

\end{enumerate}



\end{document}
